\documentclass[12pt]{article}
\usepackage[utf8]{inputenc}
\usepackage[T1]{fontenc}
\usepackage{amsmath,amsfonts,amssymb}
\usepackage{graphicx}
\usepackage{a4wide}

\begin{document}
\title{EEG decoding}
\author{Vadim Strijov\footnote{E-mail: vadim@m1p.org}}
\maketitle

\begin{abstract}
\end{abstract}
\noindent \textbf{Keywords:} signal processing
 
See folder /Users/victor/tmp/ArXiv/ArXiV2024/ EEGinv and CCM
 
\section{Introduction}


From~\cite{LopezRincon2016} highlights: The EEG inverse problem is solved using the bidomain model. A spatial comparison is made with fMRI using the linkRBrain platform.\cite{LopezRincon2016}

From~\cite{Yang2021} highlights (Joyitta Dutta): The spatiotemporal trajectory of tau.  An analytic graph diffusion framework can model tau progression along the structural connectome. The combination of longitudinal tau positron emission tomography and diffusion tensor imaging offers a macroscopic perspective on tau propagation.

Makaroff~\cite{Makaroff2023} inverse EEG problem

An attempt to localize brain electrical activity sources using EEG with limited
number of electrodes:~\cite{Majkowski2016} LORETA algotithm, 2016

\subsection{CCM}
CCM and MI for EEG~\cite{Ge2022}  Symbolic Convergent Cross Mapping Based on Permutation Mutual Information

CCM for ECG with Gaussian processes and Kernels \cite{Feng2020}

(not important) sugihara on S-map coefficients and interaction strength~\cite{Munch2022}\cite{Sugihara2012}

(not related but useful explanation and ccm criterions) Sugihara in Science 2012, explanations of CCM in the Supplimentary mayerials~\cite{Sugihara2012} and here many plots~\cite{Ye2016} in the supp mat. (btw it titles Overcoming the curse of dimensionality)




\subsection{CCA}
PLS and autoencoder and LSTM for BCI ECoG \cite{Ran2022}



Where are the fMRI + EEG???


\subsection{data}
\cite{Berezutskaya2022}\cite{Musk2019}

\bibliographystyle{unsrt}
\bibliography{EEGdecodeReview}
\end{document}